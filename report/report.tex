\documentclass{article}


% if you need to pass options to natbib, use, e.g.:
%     \PassOptionsToPackage{numbers, compress}{natbib}
% before loading neurips_2023


% ready for submission
% \usepackage{neurips_2023}


% to compile a preprint version, e.g., for submission to arXiv, add add the
% [preprint] option:
    \usepackage[preprint]{neurips_2023}


% to compile a camera-ready version, add the [final] option, e.g.:
    % \usepackage[final]{neurips_2023}


% to avoid loading the natbib package, add option nonatbib:
  %  \usepackage[nonatbib]{neurips_2023}


\usepackage[utf8]{inputenc} % allow utf-8 input
\usepackage[T1]{fontenc}    % use 8-bit T1 fonts
\usepackage{hyperref}       % hyperlinks
\usepackage{url}            % simple URL typesetting
\usepackage{booktabs}       % professional-quality tables
\usepackage{amsfonts}       % blackboard math symbols
\usepackage{nicefrac}       % compact symbols for 1/2, etc.
\usepackage{microtype}      % microtypography
\usepackage{xcolor}         % colors


\title{BME1312 Project1 Report}


% The \author macro works with any number of authors. There are two commands
% used to separate the names and addresses of multiple authors: \And and \AND.
%
% Using \And between authors leaves it to LaTeX to determine where to break the
% lines. Using \AND forces a line break at that point. So, if LaTeX puts 3 of 4
% authors names on the first line, and the last on the second line, try using
% \AND instead of \And before the third author name.


\author{%
  Wenye Xiong \\
  2023533 \\
  \texttt{xiongwy2023@shanghaitech.edu.cn}
  \And
  Renyi Yang \\
  2023533030 \\
  \texttt{yangry2023@shanghaitech.edu.cn}
  \AND
  Jiaxing Wu \\
  2023533 \\
  \texttt{wujx2023@shanghaitech.edu.cn}
  \And
  Boyang Xia \\
  2023533 \\
  \texttt{xiaby2023@shanghaitech.edu.cn}
  \AND
  Fengmin Yang \\
  2023533 \\
  \texttt{yangfm2023@shanghaitech.edu.cn}
}

\begin{document}


\maketitle


\begin{abstract}
  A project about MRI Image Reconstruction.
  Reconstruct high-quality MRI images from undersampled k-space data.

  Implement a U-Net and a 3D residual network with dropout,
  dynamic learning rate scheduling and data augmentation to optimize performance.
\end{abstract}

\section{Model}




\section{Tips for writing}
\label{tips}


\subsection{Headings: second level}
\subsubsection{Headings: third level}



\subsection{Footnotes}

\footnote{As in this example.}

\subsection{Figures}

% \begin{figure}
%   \centering
%   \fbox{\rule[-.5cm]{0cm}{4cm} \rule[-.5cm]{4cm}{0cm}}
%   \caption{Sample figure caption.}
% \end{figure}


\subsection{Tables}

% \begin{table}
%   \caption{Sample table title}
%   \label{sample-table}
%   \centering
%   \begin{tabular}{lll}
%     \toprule
%     \multicolumn{2}{c}{Part}                   \\
%     \cmidrule(r){1-2}
%     Name     & Description     & Size ($\mu$m) \\
%     \midrule
%     Dendrite & Input terminal  & $\sim$100     \\
%     Axon     & Output terminal & $\sim$10      \\
%     Soma     & Cell body       & up to $10^6$  \\
%     \bottomrule
%   \end{tabular}
% \end{table}


%%%%%%%%%%%%%%%%%%%%%%%%%%%%%%%%%%%%%%%%%%%%%%%%%%%%%%%%%%%%


\end{document}